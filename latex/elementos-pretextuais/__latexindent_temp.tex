% ----------------------------------------------------------
% AGRADECIMENTOS
% ----------------------------------------------------------
\begin{agradecimentos}
Os agradecimentos principais são direcionados à Gerald Weber, Miguel Frasson,
Leslie H. Watter, Bruno Parente Lima, Flávio de Vasconcellos Corrêa, Otávio Real
Salvador, Renato Machnievscz\footnote{Os nomes dos integrantes do primeiro
projeto abn\TeX\ foram extraídos de
\url{http://codigolivre.org.br/projects/abntex/}} e todos aqueles que
contribuíram para que a produção de trabalhos acadêmicos conforme
as normas ABNT com \LaTeX\ fosse possível.

As melhores formas de começar um texto. São possível com a assim o melhor j

Eu não conheço outra metodologia do que a explicitada pelo professor Ananias. É uma forma inteligente de alcançar o sucesso.
Assim vamos começar o que eu melhor conheço como ideas inatas.

Assim é conhecido o objeto desse texto

Assim eu não quero mais conhecer o que é desconhecido. O que não foi permitido conhecer ou desvendar.

A forma conhecida desta ideia pode ser e
Agradecimentos especiais são direcionados ao eu de Pesquisa em Arquitetura
da Informação\footnote{\url{http://www.cpai.unb.br/}} da Universidade de
Brasília (CPAI), ao grupo de usuários
\emph{latex-br}\footnote{\url{http://groups.google.com/group/latex-br}} e aos
novos voluntários do grupo
\emph{\abnTeX}\footnote{\url{http://groups.google.com/group/abntex2} e
\url{http://abntex2.googlecode.com/}}~que contribuíram e que ainda
contribuirão para a evolução do \abnTeX.


\end{agradecimentos}


