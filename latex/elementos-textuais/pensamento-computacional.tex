\chapter{Pensamento Computacional}\label{pensamento-computacional}

% A evolução da computação nos últimos anos, aliado ao seu barateamento, tem tornado possível o desenvolvimento de novas estratégias para resolução de problemas. Fato que tem produzido impactos notáveis tanto na academia quanto nas empresas.

% barateamento de capacidades computacionais de armazenamento e processamento

% As últimas 2 ou 3 décadas dominado - e o mundo transformado - pelo advento de computação cada vez mais poderosa, menos dispendiosa e onipresente, além do surgimento da World Wide Web e tecnologias relacionadas.


A evolução da computação nas últimas décadas, aliada ao seu barateamento, tem produzido impactos notáveis tanto na academia quanto na industria. A disponibilidade de dispositivos mais baratos e com maior capacidade de processamento e armazenamento tem tornado a computação ubiqua. 

% **Enriquecimento das formas de exploração de fenômenos.**

Na academia, esse fenômeno tem favorecido o surgimento de novas estratégias para exploração de fenômenos. Até meados do século 20, todo progresso cientifico foi conduzido apenas por interações entre atividades experimentais e analíticas\footnote{
Em sentido mais amplo, quando mencionamos `atividade analítica' estamos nos referindo à utilização de aparato matemático para resolução teórica de problemas. Ao mesmo tempo, o termo análise também pode fazer referência ao emprego da análise matemática, ramo da matemática que lida com conceitos do cálculo diferencial, tais como diferenciação, integração, e séries infinitas.}. O surgimento da computação e o seu desenvolvimento desde então trouxe consigo novas formas de fazer ciência, tais como a simulação e modelagem computacional bem como, mais recentemente, a mineração de dados e o aprendizado de máquina, úteis para a análise de grande volume de informação. %TODO (cite weintrop, and wing)



com destaque para simulação numérica que permite que fenômenos muito complexos sejam analiticamente tratáveis.

As vantagens do emprego de simulações numéricas podem ser exemplificados com o problemas 




%Quando nos mencionamos `atividade analítica' (trabalho analítico) estamos nos referindo às contribuição da análise matemática matematica (analise matematica) que invove limties, tais como diferenciação, integração, sequecia infinitas. Em um sentido mais amplos expressão analise também pode ser usada simples resolução de formulas usando manipulação algebrica, o que é fetio para diferenciar-se da geometria. (MELHORAR)}.

% Outra idéia levemente provocativa é que a ciência da computação aplicada está desempenhando o papel que a matemática fez do século XVII ao século XX: fornecer uma estrutura ordenada e formal e um aparato exploratório para outras ciências. Além de seu aparentemente feliz caso com a teoria das cordas, é difícil dizer o que a matemática está fazendo para outras ciências hoje; A maioria dos cientistas de matemática que utilizamos hoje foi desenvolvida há mais de um século.

% Recente avanços na computação de alta velocidade e em metdos analiticos criou ferramentas poderas para entender fenomenos em todos os espectros do pensamento humano (human inquiry)

%Wing(2006) argumenta que o pensamento computacional se tornaria fundamental pra todas as disicplinas que os avanços na computação habilitaria pesquisadores para desenvolver (envision) novas estrategias de resolução de problemas e teste de novas soluções.

%Todo o desenvolvimento científico até meados do século 20 se deu a partir do uso atividades analiticas e experimentais.

%**Simulação e modelagem computacional**

