\chapter{Pensamento Computacional}\label{pensamento-computacional}


\section{Histórico e definição}

A evolução da computação nos últimos anos, aliado ao seu barateamento, tem tornado possível o desenvolvimento de novas estratégias para resolução de problemas, o que tem produzido impactos notáveis tanto na academia quanto nas empresas.

O aprofundamento do entendimento de sistemas orgânicos, por exemplo, trazido pelo aperfeiçoamento dos métodos de modelagem computacional, tanto quanto a coleta e a análise massiva de dados pelas empresas com o fim de investigar comportamentos de consumo, ilustram esse fenômeno.

Essencialmente, o que desenvolvimento recente da computação tem facultado é a otimização de métodos que permitem a exploração de modelos dinâmicos, ou não-determinísticos, sejam eles o avanço de doenças, mudança climática ou simplesmente cliques ou curtidas em redes sociais \cite{weintrop}.

Alguns desdobramentos desse fenômeno também são visíveis no nosso cotidiano. Ao utilizarmos um mecanismo de busca, ou quando utilizamos um tradutor, estamos nos apoiando em algoritmos de coleta e análises de dados que dotam computadores, literalmente, com a capacidade aprender -- processo conhecido como aprendizado de máquina (em inglês: ``machine learning''). Algumas projeções revelam que em poucas décadas um número expressivo de atividades laborais serão simplesmente substituídas por inteligência artificial. %TODO: Procurar referência

No ano de 2006, refletindo sobre as novas práticas de resolução de problemas trazidas pela evolução da capacidade e dos métodos computacionais, a professora da Universidade de Columbia Jeannette Wing, em um artigo seminal, formaliza o conceito de \textbf{pensamento computacional}.

% A idea de um "pensamento" computacional, busca, no entender da autora, abranger um conjunto de habilidades

Conforme sintetiza em um outro artigo, no ano de 2010,

% TODO: citar frase seminal (``[...] draws on [...]'')

% TODO: trazer elementos da ciência da computação para discussão, mostrando o seu potencial para resolver problemas (usar tabela comparativa se possível).

% TODO: para reforçar a importância da ciência da computação nas mais diversas áreas, citar a adoção de curso de programação em cursos de graduação diferentes da ciência computação.

% TODO: citar como exemplo da relevância da forma de pensar computacionalmente, a contratação em massa de mestres e doutores Ciência computação em wall street. Usar https://insights.dice.com/2017/11/20/got-phd-cs-statistics-consider-wall-street/ como referência 

% TODO: citar exemplos apresentados pela autora de possíveis usos de um "pensamento" computacional em tarefas do dia-dia (Trata-se de uma argumentação fraca, mas legítima. Apenas considerar.)

\begin{citacao}
  O pensamento computacional $[$...$]$ descreve a atividade mental na formulação de um problema para admitir 
uma solução computacional. A solução pode ser realizada por um humano ou máquina, 
ou mais geralmente, por combinações de seres humanos e máquinas \cite[tradução nossa]{Wing2010}.
\end{citacao}



% Para a autora, a ideia de uma "pensamento" ou simplesmente de uma "abordagem" computacional para resolução problemas, não seria apenas relevante para programadores ou cientistas da computação, 

% Em essência, o que se busca compor uma abordagem para resolução de problemas, este conceito e propõe a abranger um conjunto de habilidades e práticas não apenas relevante para programadores ou cientistas da computação, mas passível de compor até mesmo as faculdades analíticas de qualquer criança \cite{wing2006}.






% O uso da palavra "computacional" busca também evidenciar a relação dessa abordagem 
% com elementos e temas próprios da ciência da computação, tais como a confecção de algorítimos (redação de passos para resolver um problemas), 
