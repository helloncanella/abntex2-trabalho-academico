\chapter{Pensamento Computacional}\label{pensamento-computacional}

% A evolução da computação nos últimos anos, aliado ao seu barateamento, tem tornado possível o desenvolvimento de novas estratégias para resolução de problemas. Fato que tem produzido impactos notáveis tanto na academia quanto nas empresas.

% barateamento de capacidades computacionais de armazenamento e processamento

% As últimas 2 ou 3 décadas dominado - e o mundo transformado - pelo advento de computação cada vez mais poderosa, menos dispendiosa e onipresente, além do surgimento da World Wide Web e tecnologias relacionadas.


A evolução da computação nas últimas décadas, aliada ao seu barateamento, tem produzido impactos notáveis tanto na academia quanto na industria. A disponibilidade de dispositivos mais baratos e com maior capacidade de processamento e armazenamento tem tornado a computação ubiqua. 

% **Enriquecimento das formas de exploração de fenômenos.**

Na academia, esse fenômeno tem favorecido o surgimento de novas estratégias para exploração de fenômenos. Até meados do século 20, todo progresso cientifico foi conduzido apenas por interações entre atividades experimentais e analíticas\footnote{
Em sentido mais amplo, quando mencionamos `atividade analítica' estamos nos referindo à utilização de aparato matemático para resolução teórica de problemas. Ao mesmo tempo, o termo análise também pode fazer referência ao emprego da análise matemática, ramo da matemática que lida com conceitos do cálculo diferencial, tais como diferenciação, integração, e séries infinitas.}. O surgimento da computação e o seu desenvolvimento desde então trouxe consigo novas formas de fazer ciência, tais como a simulação e modelagem computacional bem como, mais recentemente, a mineração de dados e o aprendizado de máquina, úteis para a análise de grande volume de informação. \cite{Djorgovski2005, wing2006} %TODO (cite weintrop, and wing)


O uso de simulações numéricas, por exemplo, se justifica ao permitir que um grande número de fenômenos muito complexos sejam analiticamente tratáveis. Em muitos casos é única forma de exploração possível. Mesmo na mecânica newtoniana mais simples, é possível resolver  exatamente, apenas, o problema de dois corpos. Para $N\geq3$, soluções numéricas são necessárias. Da astronomia podemos retirar alguns exemplos, como a formação de estrelas e galáxias e explosão estrelares - de modo geral, qualquer evento envolvendo turbulência.  \cite[]{Djorgovski2005} %weintrop

O uso de métodos computacionais, tal como a simulação, tem expandido a abrangência de sistemas não lineares que tem sido explorados pelo modelos matemáticos e computacionais. Como lembra \citeonline{Weintrop2016}, campos da ciência estão experimentando um renascimento de abordagens experimentais em razão do acesso facilitado a mais poder computacional. 

O autor destaca que num passado recente, para muitos pesquisadores apenas o estudo de sistemas determinísticos era viável, tendo o termo `não-linear', praticamente, o sinônimo de `insolúvel'. Havia desse modo a propensão de investigação computacional apenas de sistemas lineares. Esse quadro era especiamente verdade para muitas pesquisas em biologia e química. 

Esse processo ganha especial relevância ao nos darmos conta da natureza caótica da ampla maioria dos fenômenos físicos. Sistemas lineares e determinísticos são portanto exceções, e não a regra \cite[]{Weintrop2016}.

Outros agentes de transformações da prática científica, embora mais recentes e em fase nascente, são as novas possibilidades trazidas pela abundância e pelo barateamento do armazenamento grandes volumes de dados. Vivemos uma era onde a disponibilidade de dados gerados por câmeras, sensores, execução de simulações e registro de interação humana crescem exponencialmente. 

Nesse cenário, como ressalta \citeonline[]{Djorgovski2005}, o foco de valores tem mudado da propriedade de dados ou de instrumentos para reuni-los para a propriedade de conhecimento e ideas que tornam possíveis a extração de significados desse volume de informação. 

A abundância traz consigo muitos desafios. A taxa com que cientistas e engenheiros tem coletado e produzido dados vem exigindo avanços nas estratégias de análise. O acúmulo chegou a um nível de complexidade que, certo modo, tem sido impossível fazer qualquer tipo de investigação superficial utilizando técnicas convencionais. Lidar com esse conjunto desestruturado e rico de dados, extraindo dele significado, tem sido uma das batalhas da atual revolução científica e industrial \cite[]{Djorgovski2005}. Nesse contexto o emprego de técnicas de aprendizado de máquina é essencial.

Em linhas gerais, o aprendizado de máquina (do inglês: \textit{machine learning}) fundamenta-se no uso algoritmos que instruem computadores sobre como avaliar dados e deles extrair padrões,correlações, permitindo-os, de forma extraordinária, a fazer predições. E esse processo tem um componente recursivo: quanto mais análises são feitas, mais experiência e competência são adquiridas. Ou seja, mais `inteligentes' se tornam.

\citeonline[]{Escobar} propõe uma visão simplificada das interações humanas que facilita o entendimento desses algoritmos. Como descreve, ao conhecemos alguém pela primeira vez, baseando-nos em modelos pessoais, somos capazes de dizer nos primeiros minutos se essa pessoa nos transmite boa ou má impressão. Para cada nova pessoa que encontramos, avaliamos algumas de suas características e as registramos. Esse processo nos permite refinar e recompor modelos, que irão influenciar outras percepções em interações futuras. 

É exatamente nesse princípio recursivo que se fundamenta o aprendizado de máquina: categorizar dados de acordo com suas características com o fim de compor e refinar modelos.








% Aprendizado de máquina, ou simplesmente inteligência artificial, basea-se grosso modo 



% Outra idéia levemente provocativa é que a ciência da computação aplicada está desempenhando o papel que a matemática fez do século XVII ao século XX: fornecer uma estrutura ordenada e formal e um aparato exploratório para outras ciências. Além de seu aparentemente feliz caso com a teoria das cordas, é difícil dizer o que a matemática está fazendo para outras ciências hoje; A maioria dos cientistas de matemática que utilizamos hoje foi desenvolvida há mais de um século.

% Recente avanços na computação de alta velocidade e em metdos analiticos criou ferramentas poderas para entender fenomenos em todos os espectros do pensamento humano (human inquiry)

%Wing(2006) argumenta que o pensamento computacional se tornaria fundamental pra todas as disicplinas que os avanços na computação habilitaria pesquisadores para desenvolver (envision) novas estrategias de resolução de problemas e teste de novas soluções.

%Todo o desenvolvimento científico até meados do século 20 se deu a partir do uso atividades analiticas e experimentais.

%**Simulação e modelagem computacional**

