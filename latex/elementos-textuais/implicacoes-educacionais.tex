\chapter{Implicações para aprendizagem científica}

O desenvolvimento da computação e o amadurecimento da noção de pensamento computacional traz consigo várias implicações e oportunidades educacionais.

De partida, pode-se dizer que a efetivação do uso do computador como ferramenta para resolução de problemas exige a formação de capital humano. Normalmente, espera-se que estudantes tenham acesso à universidade para introdução das primeiras noções de programação e ciência da computação. Sabe-se contudo que eles se mostram cada vez familiarizados com \textit{smartphones} e outros dispositivo computacionais. Por isso podemos nos perguntar: por que não antecipar essa instrumentalização? 

Esse questionamento se torna ainda mais pertinente em face do percentual reduzido de portadores de grau universitário. A demanda por profissionais com experiência na resolução de problemas computacionais tem crescido numa proporção superior a que academia tem formado. Esse desequilíbrio tem levado grandes empresas de tecnologia como o Google e IBM a dispensar a formação universitária como pre-requisito para contratação \cite{Purtill}.

A motivação para essa antecipação não é apenas de caráter laboral. Como discutiremos ao longo desse capítulo o uso de computadores e do pensamento computacional oferece excelentes contextos para desenvolvimento de competências científicas. O reverso também é verdadeiro: problemas científicos e matemáticos são domínios excelentes para a aplicação e exercício do pensamento computacional. É nessa tese que esse trabalho se apoia. 

No capítulo \ref{pensamento-computacional} mostramos o como conceitos de abstração, decomposição de problemas e simulação, aliados a praticas de representação de problemas se estruturam em torno do pensamento computacional. Ao mesmo tempo que são centrais na ciência do computação, estes elementos também são fundamentais para o desenvolvimento de modelos e para a compreensão e resolução de problemas num largo espectro de disciplinas matemáticas e científicas \cite{Sengupta2013}.

% Soloway argumenta que aprender a programar corresponde a aprender construir mecanismos e explicações. Portanto a capacidade de construir modelos computacionais por programação corresponde a 

A literatura tem demonstrado diversas formas o como a aprendizagem de programação - uma das praticas de representação - em conjunto com conceitos de outros domínios pode ser mais fácil do que aprender cada um desses tópicos individualmente. Essa abordagem é destacadamente diferente da forma como o ensino de computação tem sido trabalhado no ambiente escolar, onde as atividades propostas focam em aspectos apartados da realidade.

Em um estudo citado por \citeonline{Weintrop2016}, descobriu-se que a introdução de programação no contexto da modelagem de fenômenos físicos e químicos, durante as atividades de alunos de graduação, resultaram no aprendizado mais efetivo de programação, e no aumento do engajamento com a área de domínio das tarefas.

A literatura também nos apresenta outros exemplos de contextualização de atividades computacionais com o ensino de ciências. Dentre algumas propostas práticas, podemos destacar o uso eficiente do software Netlogo \cite{netlogo} na introdução de noções de probabilidade e estatística, como relatado por \citeonline{ProbLab}, e na modelagem de fenômenos epidemiológicos, tal como reportado em \cite{Lee:2011:CTY:1929887.1929902}.

% O enriquecimento das atividades em sala dae aula não tem sido o único
Vários estudos apontam que o enriquecimento trazido por essa abordagem conjunta, associada com o caráter ``mão na massa'' de muitas dessas atividades, produz impactos positivos na forma como os alunos percebem as aulas de ciências \cite{Lee:2011:CTY:1929887.1929902, Barr2011}. A sensação de estar resolvendo um problema autêntico com real aplicabilidade e, em muitos casos, inserido na realidade do qual fazem parte, produz aumentos sensíveis dos níveis de engajamento.

\citeonline{Weintrop2016} sugere que o aumento da atratividade dessas disciplinas, apresentadas dessa forma, também pode favorecer o aumento da representação de grupos minoritários nos meios científicos, tais como mulheres, negros e transgêneros.

Como permitir que eles invertam o papel de meros consumidores de tecnologia para então também se tornarem criadores?

Como vimos no capítulo \ref{computadores-e-ciencia}, a atividade científica tem se tornado cada vez mais um empreendimento computacional. Portanto, apoiar a sua aprendizagem no desenvolvimento e análise de modelos computacionais significa apresentar aos alunos uma visão mais realista e coerente com a forma que ela é exercida profissionalmente. Em última instância, essa abordagem também pode ter impactos laborais na medida em que pode estimulá-los a perseguir uma carreira científica.















