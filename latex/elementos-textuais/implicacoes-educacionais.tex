\chapter{Implicações para aprendizagem científica}

O desenvolvimento da computação e o amadurecimento da noção de pensamento computacional traz consigo várias implicações e oportunidades educacionais.

De partida, pode-se dizer que a efetivação do uso do computador como ferramenta para resolução de problemas exige a formação de capital humano. Normalmente, espera-se que estudantes tenham acesso à universidade para introdução das primeiras noções de programação e ciência da computação. Sabe-se contudo que eles se mostram cada vez familiarizados com \textit{smartphones} e outros dispositivo computacionais. Por isso podemos nos perguntar: por que não antecipar essa instrumentalização? 

Esse questionamento se torna ainda mais pertinente em face do percentual reduzido de portadores de grau universitário. A demanda por profissionais com experiência na resolução de problemas computacionais tem crescido numa proporção superior a que academia tem formado. Esse desequilíbrio tem levado grandes empresas de tecnologia como o Google e IBM a dispensar a formação universitária como pre-requisito para contratação \cite{Purtill}.

A motivação para essa antecipação não é apenas de caráter laboral. Como discutiremos ao longo desse capítulo o uso de computadores e do pensamento computacional oferece excelentes contextos para desenvolvimento de competências científicas. É nessa possibilidade que a tese desse trabalho se apoia.

\citeonline{Sengupta2013}, 

Como vimos no capítulo \ref{computadores-e-ciencia} a atividade científica tem se tornado cada vez mais um empreendimento computacional. Portanto, apoiar a sua aprendizagem no desenvolvimento e análise de modelos computacionais significa apresentar aos alunos uma visão mais realista e coerente com a forma que ela é exercida profissionalmente. Em última instância, essa abordagem também pode ter impactos laborais na medida em que pode estimulá-los a perseguir uma carreira científica.











