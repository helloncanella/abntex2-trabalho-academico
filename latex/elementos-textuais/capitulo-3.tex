\chapter{Pensamento Computacional}\label{pensamento-computacional}

\section{Histórico e definição}

O intenso avanço observado na computação nos últimos anos tem tornado possível o desenvolvimento de novas estratégias para resolução problemas, produzindo impactos significativos tanto na academia quanto na indústria.

Com a disponibilização de poder computacional relativo baixo custo tem sido possível testar soluções através da execução de simulações e coleta massiva de dados, por exemplo, 

Em um artigo seminal, no ano 2006,  a professora da Universidade de Columbia Jeannette Wing, formaliza a noção de ``Pensamento Computational". Entendido como a uma abordagem para resolução de problemas, este conceito e propõe a abranger um conjunto de habilidades e práticas não apenas relevante para programadores ou cientistas da computação, mas passível de compor até mesmo as faculdades analíticas de qualquer criança \cite{wing2006}.

Em linhas gerais, conforme sintetiza em artigo de 2010, 
  
\begin{citacao}
  (...) descreve a atividade mental na formulação de um problema para admitir 
uma solução computacional. A solução pode ser realizada por um humano ou máquina, 
ou mais geralmente, por combinações de seres humanos e máquinas \cite[tradução nossa]{Wing2010}.
\end{citacao}



% O uso da palavra "computacional" busca também evidenciar a relação dessa abordagem 
% com elementos e temas próprios da ciência da computação, tais como a confecção de algorítimos (redação de passos para resolver um problemas), 
