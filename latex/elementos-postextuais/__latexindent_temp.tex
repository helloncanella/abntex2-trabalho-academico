%% This BibTeX bibliography file was created using BibDesk.
%% http://bibdesk.sourceforge.net/


@book{Locke,
	author = {Locke, J.},
	year={1690/1979},
	title = {An essay concerning human understanding.},
	publisher = {Oxford Univerty Press},
	address = {New York, NY, USA},
}

Wilensky U (1999b) NetLogo. Center for Connected Learning and Computer-Based Modeling, Northwestern University, Evanston. http://ccl.northwestern.edu/netlogo

@article{Harel,
	author = { Idit   Harel  and  Seymour   Papert },
	title = {Software Design as a Learning Environment},
	journal = {Interactive Learning Environments},
	volume = {1},
	number = {1},
	pages = {1-32},
	year  = {1990},
	publisher = {Routledge},
	doi = {10.1080/1049482900010102},
	URL = { 
		https://doi.org/10.1080/1049482900010102
	},
	eprint = { 
		https://doi.org/10.1080/1049482900010102
	}
}

@article{ProbLab,
abstract = {ProbLab, an experimental middle-school unit in probability and statistics, includes a
suite of computer-based interactive models authored in NetLogo (Wilensky, 1999). We
explain the rationale of two of the models, Stochastic Patchwork and Sample Stalagmite,
and their potential as learning supports, e.g., the temporal–spatial metaphor: sequences
of stochastic events (occurring over time) are grouped as arrays (laid out in space) that
afford proportional judgment. We present classroom episodes that demonstrate how the
Law of Large Numbers (many samples) can be mapped onto the classroom social space
(many students) as a means of facilitating discussion and data sharing and
contextualizing the content. We conclude that it is effective to embed the Law of Large
Social Numbers into designs for collaborative learning of probability and statistics},
author = {Dor Abrahamson and Uri Wilensky},
title = {{ Problab goes to school: design, teaching, and
learning of probability with multi-agent
interactive computer models}},
url = {https://ccl.northwestern.edu/2005/Abr+Wil_CERME4.pdf},
year = {2005}
}

@article{Lee:2011:CTY:1929887.1929902,
 author = {Lee, Irene and Martin, Fred and Denner, Jill and Coulter, Bob and Allan, Walter and Erickson, Jeri and Malyn-Smith, Joyce and Werner, Linda},
 title = {Computational Thinking for Youth in Practice},
 journal = {ACM Inroads},
 issue_date = {March 2011},
 volume = {2},
 number = {1},
 month = feb,
 year = {2011},
 issn = {2153-2184},
 pages = {32--37},
 numpages = {6},
 url = {http://doi.acm.org/10.1145/1929887.1929902},
 doi = {10.1145/1929887.1929902},
 acmid = {1929902},
 publisher = {ACM},
 address = {New York, NY, USA},
 keywords = {abstraction, analysis, automation, computational thinking, computer science education},
} 



@article{Piwek2016,
abstract = {You will learn about algorithms and abstraction in this free course, Introduction to computational thinking, and encounter some applications of computational thinking in various disciplines, ranging from biology and physics to economics and sport science. After studying this course, you should be able to describe the skills that are involved in computational thinking, define and use the concepts of abstraction as modelling and abstraction as encapsulation, understand the distinctive nature of computational thinking, when compared with engineering and mathematical thinking, be aware of a range of applications of computational thinking in different disciplines.},
author = {Piwek, P},
title = {{Introduction to Computational Thinking}},
url = {http://doer.col.org/handle/123456789/6196},
year = {2016}
}

@Misc{netlogo,
author =   {Uri Wilensky},
title =    {Netlogo},
howpublished = {http://ccl.northwestern.edu/netlogo/},
year = {1999}
}

@book{Gough,
	author = {John Gough, W. Atkinson},
	year={1813},
	title = {Practical Arithmetic: In Four Books},
}

@article{Sengupta2013,
abstract = {Computational thinking (CT) draws on concepts and practices that are fundamental to computing and computer science. It includes epistemic and representational practices, such as problem representation, abstraction, decomposition, simulation, verification, and prediction. However, these practices are also central to the development of expertise in scientific and mathematical disciplines. Recently, arguments have been made in favour of integrating CT and programming into the K-12 STEM curricula. In this paper, we first present a theoretical investigation of key issues that need to be considered for integrating CT into K-12 science topics by identifying the synergies between CT and scientific expertise using a particular genre of computation: agent-based computation. We then present a critical review of the literature in educational computing, and propose a set of guidelines for designing learning environments on science topics that can jointly foster the development of computational thinking with scientific expertise. This is followed by the description of a learning environment that supports CT through modeling and simulation to help middle school students learn physics and biology. We demonstrate the effectiveness of our system by discussing the results of a small study conducted in a middle school science classroom. Finally, we discuss the implications of our work for future research on developing CT-based science learning environments. {\textcopyright} 2012 Springer Science+Business Media New York.},
author = {Sengupta, Pratim and Kinnebrew, John S. and Basu, Satabdi and Biswas, Gautam and Clark, Douglas},
doi = {10.1007/s10639-012-9240-x},
isbn = {9789898565068},
issn = {13602357},
journal = {Education and Information Technologies},
keywords = {Agent-based modeling and simulation,Biology education,Computational modeling,Computational thinking,Learning by design,Multi-agent systems,Physics education,Science education,Visual programming},
number = {2},
pages = {351--380},
title = {{Integrating computational thinking with K-12 science education using agent-based computation: A theoretical framework}},
volume = {18},
year = {2013}
}


@book{Papert1983,
	abstract = {Mindstorms has two central themes: that children can learn to use computers in a masterful way and that learning to use computers can change the way they learn everything else. Even outside the classroom, Papert had a vision that the computer could be used just as casually and as personally for a diversity of purposes throughout a persons entire life. Seymour Papert makes the point that in classrooms saturated with technology there is actually more socialization and that the technology often contributes to greater interaction among students and among students and instructors.},
	author = {Papert, Seymour},
	booktitle = {New Ideas in Psychology},
	pages = {87},
	title = {{Mindstorms: Children, computers and powerful ideas}},
	volume = {1},
	year = {1983}
}

@ONLINE{Hatch,
	author = {Hatch, Robert A.},
	title = {Ptolemy's planetary models},
	year = {1998},
	Urlaccessdate={11 de Setembro de 2018},
	url = {http://users.clas.ufl.edu/ufhatch/pages/03-Sci-Rev/SCI-REV-Home/resource-ref-read/chief-systems/08-0PTOL3-WSYS.html}
}
@ONLINE{Purtill,
	author = {Purtill, Corinne},
	title = {Apple, IBM, and Google don’t care anymore if you went to college},
	day = {23},
	month = {Aug},
	year = {2018},
	Urlaccessdate={12 de Outubro de 2018},
	url = {https://qz.com/work/1367191/apple-ibm-and-google-dont-require-a-college-degree/}
}

@ONLINE{Bloomberg,
	author = {Verhage, Julie},
	title = {This Robot Said to Sell Facebook. Next Time It May Be Right},
	day = {21},
	month = {Nov.},
	year = {2017},
	Urlaccessdate={11 de Setembro de 2018},
	url = {https://www.bloomberg.com/news/articles/2017-11-21/this-robot-said-to-sell-facebook-next-time-it-may-be-right}
}

@ONLINE{biocomputer,
	author = {Graham, Ringo},
	title = {How MIT’s new biological ‘computer’ works, and what it could do in the future},
	month = {07},
	year = {2016},
	Urlaccessdate={29 de julho de 2018},
	url = {https://www.extremetech.com/extreme/232190-how-mits-new-biological-computer-works-and-what-it-could-do-in-the-future}
}

@ONLINE{Escobar,
	author = {Escobar, Matt},
	title = {Artificial intelligence: here’s what you need to know to understand how machines learn},
	day = {22},
	month = {02},
	year = {2017},
	Urlaccessdate={10 de Setembro de 2018},
	url = {https://theconversation.com/artificial-intelligence-heres-what-you-need-to-know-to-understand-how-machines-learn-72004}
}

@ONLINE{Vutha,
	author = {Vutha, Amar},
	title = {Could machine learning mean the end of understanding in science?},
	day = {2},
	month = {08},
	year = {2018},
	Urlaccessdate={10 de Setembro de 2018},
	url = {https://theconversation.com/could-machine-learning-mean-the-end-of-understanding-in-science-98995}
}

@article{PhysRevLett.120.024102,
  title = {Model-Free Prediction of Large Spatiotemporally Chaotic Systems from Data: A Reservoir Computing Approach},
  author = {Pathak, Jaideep and Hunt, Brian and Girvan, Michelle and Lu, Zhixin and Ott, Edward},
  journal = {Phys. Rev. Lett.},
  volume = {120},
  issue = {2},
  pages = {024102},
  numpages = {5},
  year = {2018},
  month = {Jan},
  publisher = {American Physical Society},
  doi = {10.1103/PhysRevLett.120.024102},
  url = {https://link.aps.org/doi/10.1103/PhysRevLett.120.024102}
}


@article{Djorgovski2005,
abstract = {All sciences, including astronomy, are now entering the era of information abundance. The exponentially increasing volume and complexity of modern data sets promises to transform the scientific practice, but also poses a number of common technological challenges. The Virtual Observatory concept is the astronomical community's response to these challenges: it aims to harness the progress in information technology in the service of astronomy, and at the same time provide a valuable testbed for information technology and applied computer science. Challenges broadly fall into two categories: data handling (or "data farming"), including issues such as archives, intelligent storage, databases, interoperability, fast networks, etc., and data mining, data understanding, and knowledge discovery, which include issues such as automated clustering and classification, multivariate correlation searches, pattern recognition, visualization in highly hyperdimensional parameter spaces, etc., as well as various applications of machine learning in these contexts. Such techniques are forming a methodological foundation for science with massive and complex data sets in general, and are likely to have a much broather impact on the modern society, commerce, information economy, security, etc. There is a powerful emerging synergy between the computationally enabled science and the science-driven computing, which will drive the progress in science, scholarship, and many other venues in the 21st century.},
archivePrefix = {arXiv},
arxivId = {astro-ph/0504651},
author = {Djorgovski, S.G.},
doi = {10.1109/CAMP.2005.53},
eprint = {0504651},
file = {:Users/helloncanella/Desktop/0504651.pdf:pdf},
isbn = {0-7695-2255-6},
journal = {Seventh International Workshop on Computer Architecture for Machine Perception (CAMP'05)},
pages = {125--132},
primaryClass = {astro-ph},
title = {{Virtual Astronomy, Information Technology, and the New Scientific Methodology}},
url = {http://ieeexplore.ieee.org/document/1508175/},
year = {2005}
}

@article{Weintrop2016,
abstract = {Science and mathematics are becoming com- putational endeavors. This fact is reflected in the recently released Next Generation Science Standards and the deci- sion to include ‘‘computational thinking'' as a core scien- tific practice. With this addition, and the increased presence of computation in mathematics and scientific contexts, a new urgency has come to the challenge of defining com- putational thinking and providing a theoretical grounding for what form it should take in school science and math- ematics classrooms. This paper presents a response to this challenge by proposing a definition of computational thinking for mathematics and science in the form of a taxonomy consisting of four main categories: data prac- tices, modeling and simulation practices, computational problem solving practices, and systems thinking practices. In formulating this taxonomy, we draw on the existing computational thinking literature, interviews with mathe- maticians and scientists, and exemplary computational {\&} David Weintrop dweintrop@u.northwestern.edu 1 thinking instructional materials. This work was undertaken as part of a larger effort to infuse computational thinking into high school science and mathematics curricular materials. In this paper, we argue for the approach of embedding computational thinking in mathematics and science contexts, present the taxonomy, and discuss how we envision the taxonomy being used to bring current educational efforts in line with the increasingly computa- tional nature of modern science and mathematics.},
author = {Weintrop, David and Beheshti, Elham and Horn, Michael and Orton, Kai and Jona, Kemi and Trouille, Laura and Wilensky, Uri},
doi = {10.1007/s10956-015-9581-5},
file = {:Users/helloncanella/Library/Application Support/Mendeley Desktop/Downloaded/Weintrop et al. - 2016 - Defining Computational Thinking for Mathematics and Science Classrooms(2).pdf:pdf},
isbn = {1059-0145},
issn = {15731839},
journal = {Journal of Science Education and Technology},
keywords = {Computational problem solving,Computational thinking,High school mathematics and science education,Modeling and simulation,STEM education,Scientific practices,Systems thinking},
number = {1},
pages = {127--147},
publisher = {Springer Netherlands},
title = {{Defining Computational Thinking for Mathematics and Science Classrooms}},
volume = {25},
year = {2016}
}



@article{wing2006,
	journal = {Commun. ACM},
	author = {Wing, Jeannette},
	number = {3},
	pages = {33--35},
	publisher = {ACM},
	title = {{Computational Thinking}},
	volume = {49},
	year = {2006},
	address = {New York, NY, USA},
}


@article{wing2008,
	author = {Wing, Jeannette},
	year = {2008},
	month = {11},
	pages = {3717-25},
	title = {Computational thinking and thinking about computing},
	volume = {366},
	booktitle = {Philosophical transactions. Series A, Mathematical, physical, and engineering sciences}
}



@article{Wing2010,
	abstract = {In my March 2006 CACM article I used the term " computational thinking " to articulate a vision that everyone, not just those who major in computer science, can benefit from thinking like a computer scientist [Wing06]. So, what is computational thinking? Here is a definition that Jan Cuny of the National Science Foundation, Larry Snyder of the University of Washington, and I use; it is inspired by an email exchange I had with Al Aho of Columbia University: Computational Thinking is the thought processes involved in formulating problems and their solutions so that the solutions are represented in a form that can be effectively carried out by an information-processing agent [CunySnyderWing10] Informally, computational thinking describes the mental activity in formulating a problem to admit a computational solution. The solution can be carried out by a human or machine, or more generally, by combinations of humans and machines. When I use the term computational thinking, my interpretation of the words " problem " and " solution " is broad; in particular, I mean not just mathematically well-defined problems whose solutions are completely analyzable, e.g., a proof, an algorithm, or a program, but also real-world problems whose solutions might be in the form of large, complex software systems. Thus, computational thinking overlaps with logical thinking and systems thinking. It includes algorithmic thinking and parallel thinking, which in turn engage other kinds of thought processes, e.g., compositional reasoning, pattern matching, procedural thinking, and recursive thinking. Computational thinking is used in the design and analysis of problems and their solutions, broadly interpreted. The most important and high-level thought process in computational thinking is the abstraction process. Abstraction is used in defining patterns, generalizing from instances, and parameterization. It is used to let one object stand for many. It is used to capture essential properties common to a set of objects while hiding irrelevant distinctions among them. For example, an algorithm is an abstraction of a process that takes inputs, executes a sequence of steps, and produces outputs to satisfy a desired goal. An abstract data type defines an abstract set of values and operations for manipulating those values, hiding the actual representation of the values from the user of the abstract data type. Designing efficient algorithms inherently involves designing abstract data types. Abstraction gives us the power to scale and deal with complexity.},
	author = {Wing, Jeannette},
	journal = {thelink - The Magaizne of the Carnegie Mellon University School of Computer Science},
	pages = {1--6},
	title = {{Computational Thinking: What and Why?}},
	url = {http://www.cs.cmu.edu/link/research-notebook-computational-thinking-what-and-why},
	year = {2010}
}

@article{weintrop,
	abstract = {Science and mathematics are becoming com- putational endeavors. This fact is reflected in the recently released Next Generation Science Standards and the deci- sion to include ‘‘computational thinking'' as a core scien- tific practice. With this addition, and the increased presence of computation in mathematics and scientific contexts, a new urgency has come to the challenge of defining com- putational thinking and providing a theoretical grounding for what form it should take in school science and math- ematics classrooms. This paper presents a response to this challenge by proposing a definition of computational thinking for mathematics and science in the form of a taxonomy consisting of four main categories: data prac- tices, modeling and simulation practices, computational problem solving practices, and systems thinking practices. In formulating this taxonomy, we draw on the existing computational thinking literature, interviews with mathe- maticians and scientists, and exemplary computational {\&} David Weintrop dweintrop@u.northwestern.edu 1 thinking instructional materials. This work was undertaken as part of a larger effort to infuse computational thinking into high school science and mathematics curricular materials. In this paper, we argue for the approach of embedding computational thinking in mathematics and science contexts, present the taxonomy, and discuss how we envision the taxonomy being used to bring current educational efforts in line with the increasingly computa- tional nature of modern science and mathematics.},
	author = {Weintrop, David and Beheshti, Elham and Horn, Michael and Orton, Kai and Jona, Kemi and Trouille, Laura and Wilensky, Uri},
	doi = {10.1007/s10956-015-9581-5},
	isbn = {1059-0145},
	issn = {15731839},
	journal = {Journal of Science Education and Technology},
	keywords = {Computational problem solving,Computational thinking,High school mathematics and science education,Modeling and simulation,STEM education,Scientific practices,Systems thinking},
	number = {1},
	pages = {127--147},
	publisher = {Springer Netherlands},
	title = {{Defining Computational Thinking for Mathematics and Science Classrooms}},
	volume = {25},
	year = {2016}
}



%% OLD

@book{ibge1993,
	Address = {Rio de Janeiro},
	Author = {IBGE},
	Date-Added = {2013-08-21 13:56:10 +0000},
	Date-Modified = {2013-08-21 13:56:10 +0000},
	Edition = {3},
	Organization = {http://biblioteca.ibge.gov.br/visualizacao/livros/liv23907.pdf},
	Publisher = {Centro de Documenta{\c c}{\~a}o e Dissemina{\c c}{\~a}o de Informa{\c c}{\~o}es. Funda{\c c}{\~a}o Intituto Brasileiro de Geografia e Estat{\'\i}stica},
	Title = {Normas de apresenta{\c c}{\~a}o tabular},
	Urlaccessdate = {21 ago 2013},
	Year = {1993}}

@misc{abntex2-wiki-como-customizar,
	Author = {abnTeX2},
	Date-Added = {2013-03-23 21:39:21 +0000},
	Date-Modified = {2013-03-23 21:44:20 +0000},
	Howpublished = {Wiki do abnTeX2},
	Keywords = {wiki},
	Title = {Como customizar o abnTeX2},
	Url = {https://code.google.com/p/abntex2/wiki/ComoCustomizar},
	Urlaccessdate = {23 mar. 2013},
	Year = {2013},
	Bdsk-Url-1 = {https://code.google.com/p/abntex2/wiki/ComoCustomizar}}

@manual{talbot2012,
	Author={Hellon Canella},
	Date-Added = {2013-03-11 12:06:04 +0000},
	Date-Modified = {2013-03-11 12:06:56 +0000},
	Month = {Nov.},
	Title = {User Manual for glossaries.sty},
	Url = {http://mirrors.ctan.org/macros/latex/contrib/glossaries/glossaries-user.pdf},
	Urlaccessdate = {11 mar. 2013},
	Year = {2012},
	Bdsk-Url-1 = {http://mirrors.ctan.org/macros/latex/contrib/glossaries/glossaries-user.pdf}}

@manual{babel,
	Author = {Johannes Braams},
	Date-Added = {2013-02-17 13:37:14 +0000},
	Date-Modified = {2013-02-17 13:38:38 +0000},
	Month = {Apr.},
	Title = {Babel, a multilingual package for use with LATEX's standard document classes},
	Url = {http://mirrors.ctan.org/info/babel/babel.pdf},
	Urlaccessdate = {17 fev. 2013},
	Year = {2008},
	Bdsk-Url-1 = {http://mirrors.ctan.org/info/babel/babel.pdf}}

@manual{abntex2modelo-artigo,
	Annote = {Este documento {\'e} derivado do \cite{abnt-bibtex-doc}},
	Author = {abnTeX2},
	Date-Added = {2013-01-15 00:10:35 +0000},
	Date-Modified = {2013-02-04 12:05:47 +0000},
	Organization = {Equipe abnTeX2},
	Title = {Modelo Can{\^o}nico de Artigo Cient{\'\i}fico com abnTeX2},
	Url = {http://abntex2.googlecode.com/},
	Year = {2013},
	Bdsk-Url-1 = {http://code.google.com/p/abntex2/}}

@manual{abntex2modelo-relatorio,
	Annote = {Este documento {\'e} derivado do \cite{abnt-bibtex-doc}},
	Author = {abnTeX2},
	Date-Added = {2013-01-15 00:05:34 +0000},
	Date-Modified = {2013-02-04 12:05:50 +0000},
	Organization = {Equipe abnTeX2},
	Title = {Modelo Can{\^o}nico de Relat{\'o}rio T{\'e}cnico e/ou Cient{\'\i}fico com abnTeX2},
	Url = {http://abntex2.googlecode.com/},
	Year = {2013},
	Bdsk-Url-1 = {http://code.google.com/p/abntex2/}}

@manual{abntex2modelo,
	Annote = {Este documento {\'e} derivado do \cite{abnt-bibtex-doc}},
	Author = {abnTeX2},
	Date-Added = {2013-01-12 22:55:32 +0000},
	Date-Modified = {2013-02-04 12:05:54 +0000},
	Organization = {Equipe abnTeX2},
	Title = {Modelo Can{\^o}nico de Trabalho Acad{\^e}mico com abnTeX2},
	Url = {http://abntex2.googlecode.com/},
	Year = {2013},
	Bdsk-Url-1 = {http://code.google.com/p/abntex2/}}

@mastersthesis{araujo2012,
	Address = {Bras{\'\i}lia},
	Author = {Lauro C{\'e}sar Araujo},
	Date-Added = {2013-01-09 11:04:42 +0000},
	Date-Modified = {2013-01-09 11:04:42 +0000},
	Month = {mar.},
	School = {Universidade de Bras{\'\i}lia},
	Subtitle = {uma perspectiva de {A}rquitetura da {I}nforma{\c c}{\~a}o da {E}scola de {B}ras{\'\i}lia},
	Title = {Configura{\c c}{\~a}o},
	Year = {2012}}

@manual{memoir,
	Address = {Normandy Park, WA},
	Author = {Peter Wilson and Lars Madsen},
	Date-Added = {2013-01-09 10:37:50 +0000},
	Date-Modified = {2013-03-21 13:23:25 +0000},
	Organization = {The Herries Press},
	Title = {The Memoir Class for Configurable Typesetting - User Guide},
	Url = {http://mirrors.ctan.org/macros/latex/contrib/memoir/memman.pdf},
	Urlaccessdate = {19 dez. 2012},
	Year = {2010},
	Bdsk-Url-1 = {http://ctan.tche.br/macros/latex/contrib/memoir/memman.pdf}}

@manual{abntex2cite-alf,
	Annote = {Este documento {\'e} derivado do \cite{abnt-bibtex-alf-doc}},
	Author = {abnTeX2 and Lauro C{\'e}sar Araujo},
	Date-Added = {2013-01-09 10:37:45 +0000},
	Date-Modified = {2013-04-05 11:03:05 +0000},
	Organization = {Equipe abnTeX2},
	Title = {O pacote abntex2cite: t{\'o}picos espec{\'\i}ficos da ABNT NBR 10520:2002 e o estilo bibliogr{\'a}fico alfab{\'e}tico (sistema autor-data)},
	Url = {http://abntex2.googlecode.com/},
	Year = {2013},
	Bdsk-Url-1 = {http://code.google.com/p/abntex2/}}

@manual{abntex2cite,
	Annote = {Este documento {\'e} derivado do \cite{abnt-bibtex-doc}},
	Author = {abnTeX2 and Lauro C{\'e}sar Araujo},
	Date-Added = {2013-01-09 10:37:45 +0000},
	Date-Modified = {2013-04-05 10:47:36 +0000},
	Organization = {Equipe abnTeX2},
	Title = {O pacote abntex2cite: Estilos bibliogr{\'a}ficos compat{\'\i}veis com a ABNT NBR 6023},
	Url = {http://abntex2.googlecode.com/},
	Year = {2013},
	Bdsk-Url-1 = {http://code.google.com/p/abntex2/}}

@manual{abntex2classe,
	Author = {abnTeX2 and Lauro C{\'e}sar Araujo},
	Date-Added = {2013-01-09 10:37:38 +0000},
	Date-Modified = {2013-04-05 11:03:48 +0000},
	Organization = {Equipe abnTeX2},
	Title = {A classe abntex2: Modelo can{\^o}nico de trabalhos acad{\^e}micos brasileiros compat{\'\i}vel com as normas ABNT NBR 14724:2011, ABNT NBR 6024:2012 e outras},
	Url = {http://abntex2.googlecode.com/},
	Year = {2013},
	Bdsk-Url-1 = {http://code.google.com/p/abntex2/}}

@manual{NBR10520:2002,
	Address = {Rio de Janeiro},
	Date-Added = {2012-12-15 21:43:38 +0000},
	Date-Modified = {2013-01-12 22:17:20 +0000},
	Month = {ago.},
	Org-Short = {ABNT},
	Organization = {Associa{\c c}\~ao Brasileira de Normas T\'ecnicas},
	Pages = 7,
	Subtitle = {Informa{\c c}\~ao e documenta{\c c}\~ao --- Apresenta{\c c}\~ao de cita{\c c}\~oes em documentos},
	Title = {{NBR} 10520},
	Year = 2002}

@manual{NBR6024:2012,
	Address = {Rio de Janeiro},
	Date-Added = {2012-12-15 21:24:06 +0000},
	Date-Modified = {2012-12-15 21:24:28 +0000},
	Month = {fev.},
	Org-Short = {ABNT},
	Organization = {Associa{\c c}\~ao Brasileira de Normas T\'ecnicas},
	Pages = 4,
	Subtitle = {Numera{\c c}\~ao progressiva das se{\c c}\~oes de um documento},
	Title = {{NBR} 6024},
	Year = 2012}

@manual{NBR6028:2003,
	Address = {Rio de Janeiro},
	Date-Added = {2012-12-15 21:02:12 +0000},
	Date-Modified = {2012-12-15 21:02:50 +0000},
	Month = {nov.},
	Org-Short = {ABNT},
	Organization = {Associa{\c c}\~ao Brasileira de Normas T\'ecnicas},
	Pages = 2,
	Subtitle = {Resumo - Apresenta{\c c}{\~a}o},
	Title = {{NBR} 6028},
	Year = 2003}

@manual{NBR14724:2001,
	Address = {Rio de Janeiro},
	Date-Added = {2012-12-15 20:34:08 +0000},
	Date-Modified = {2012-12-15 20:34:08 +0000},
	Month = {jul.},
	Org-Short = {ABNT},
	Organization = {Associa{\c c}\~ao Brasileira de Normas T\'ecnicas},
	Pages = 6,
	Subtitle = {Informa{\c c}\~ao e documenta{\c c}\~ao --- trabalhos acad\^emicos --- apresenta{\c c}\~ao},
	Title = {{NBR} 14724},
	Year = 2001}

@manual{NBR14724:2002,
	Address = {Rio de Janeiro},
	Date-Added = {2012-12-15 20:34:17 +0000},
	Date-Modified = {2012-12-15 20:34:17 +0000},
	Month = {ago.},
	Org-Short = {ABNT},
	Organization = {Associa{\c c}\~ao Brasileira de Normas T\'ecnicas},
	Pages = 6,
	Subtitle = {Informa{\c c}\~ao e documenta{\c c}\~ao --- trabalhos acad\^emicos --- apresenta{\c c}\~ao},
	Title = {{NBR} 14724},
	Year = 2002}

@manual{NBR14724:2005,
	Address = {Rio de Janeiro},
	Date-Added = {2012-12-15 20:34:08 +0000},
	Date-Modified = {2012-12-15 20:35:25 +0000},
	Month = {dez.},
	Org-Short = {ABNT},
	Organization = {Associa{\c c}\~ao Brasileira de Normas T\'ecnicas},
	Pages = 9,
	Subtitle = {Informa{\c c}\~ao e documenta{\c c}\~ao --- trabalhos acad\^emicos --- apresenta{\c c}\~ao},
	Title = {{NBR} 14724},
	Year = 2005}

@manual{NBR14724:2011,
	Address = {Rio de Janeiro},
	Date-Added = {2012-12-15 20:34:08 +0000},
	Date-Modified = {2012-12-15 20:35:25 +0000},
	Month = {mar.},
	Note = {Substitui a Ref.~\citeonline{NBR14724:2005}},
	Org-Short = {ABNT},
	Organization = {Associa{\c c}\~ao Brasileira de Normas T\'ecnicas},
	Pages = 15,
	Subtitle = {Informa{\c c}\~ao e documenta{\c c}\~ao --- trabalhos acad\^emicos --- apresenta{\c c}\~ao},
	Title = {{NBR} 14724},
	Year = 2011}

@article{van86,
	Author = {{van}, Gigch, John P. and Leo L. Pipino},
	Journal = {Future Computing Systems},
	Number = {1},
	Pages = {71-97},
	Title = {In search for a paradigm for the discipline of information systems},
	Volume = {1},
	Year = {1986}}

@phdthesis{guizzardi2005,
	Address = {Enschede, The Netherlands},
	Author = {Giancarlo Guizzardi},
	Date-Added = {2012-04-23 11:35:28 +0000},
	Date-Modified = {2012-04-23 11:35:28 +0000},
	School = {Centre for Telematics and Information Technology, University of Twente},
	Title = {Ontological Foundations for Structural Conceptual Models},
	Url = {http://www.loa.istc.cnr.it/Guizzardi/SELMAS-CR.pdf},
	Urlaccessdate = {3 jul. 2011},
	Year = {2005},
	Bdsk-Url-1 = {http://www.loa.istc.cnr.it/Guizzardi/SELMAS-CR.pdf}}

@mastersthesis{macedo2005,
	Author = {Fl{\'a}via L. Macedo},
	Date-Added = {2012-04-23 11:35:13 +0000},
	Date-Modified = {2012-04-23 11:35:13 +0000},
	Keywords = {arquitetura da informa{\c c}{\~a}o},
	School = {Universidade de Bras{\'\i}lia},
	Title = {Arquitetura da Informa{\c c}{\~a}o: aspectos espistemol{\'o}gicos, cient{\'\i}ficos e pr{\'a}ticos.},
	Type = {Disserta{\c c}{\~a}o de Mestrado},
	Year = {2005}}

@manual{EIA649B,
	Address = {EUA},
	Date-Added = {2012-04-23 11:34:59 +0000},
	Date-Modified = {2012-04-23 11:34:59 +0000},
	Keywords = {norma},
	Month = {June},
	Organization = {TechAmerica},
	Title = {ANSI/EIA 649-B: Configuration Management Standard},
	Year = {2011}}

@inproceedings{masolo2010,
	Author = {Claudio Masolo},
	Booktitle = {Proceedings of the Twelfth International Conference on the Principles of Knowledge Representation and Reasoning (KR 2010)},
	Date-Added = {2012-04-23 11:34:38 +0000},
	Date-Modified = {2012-04-23 11:34:38 +0000},
	Editor = {Lin, F. and Sattler, U.},
	Pages = {258-268},
	Publisher = {AAAI Press},
	Title = {Understanding Ontological Levels},
	Url = {http://wiki.loa-cnr.it/Papers/kr10v0.7.pdf},
	Urlaccessdate = {2 jan. 2012},
	Year = {2010},
	Bdsk-Url-1 = {http://wiki.loa-cnr.it/Papers/kr10v0.7.pdf}}

@inbook{guarino1995,
	Address = {Vienna},
	Author = {Nicola Guarino},
	Booktitle = {Philosophy and the Cognitive Science},
	Date-Added = {2012-04-23 11:34:29 +0000},
	Date-Modified = {2012-04-23 11:34:29 +0000},
	Editor = {R. Casati and B. Smith and G. White},
	Month = {Sept.},
	Pages = {443-456},
	Publisher = {Holder-Pivhler-Tempsky},
	Title = {The Ontological Level},
	Url = {http://wiki.loa-cnr.it/Papers/OntLev.pdf},
	Urlaccessdate = {2 jan. 2012},
	Year = {1995},
	Bdsk-Url-1 = {http://wiki.loa-cnr.it/Papers/OntLev.pdf}}

@incollection{bates2010,
	Address = {New York},
	Author = {Marcia J. Bates},
	Booktitle = {Encyclopedia of Library and Information Sciences},
	Date-Added = {2012-04-23 11:34:29 +0000},
	Date-Modified = {2012-04-23 11:34:29 +0000},
	Edition = {3rd},
	Editor = {Marcia J. Bates and Mary Niles Maack},
	Pages = {2347-2360},
	Publisher = {CRC Press},
	Title = {Information},
	Url = {http://pages.gseis.ucla.edu/faculty/bates/articles/information.html},
	Urlaccessdate = {24 out. 2011},
	Volume = {3},
	Year = {2010},
	Bdsk-Url-1 = {http://pages.gseis.ucla.edu/faculty/bates/articles/information.html}}

@book{doxiadis1965,
	Author = {Constantinos A. Doxiadis},
	Date-Added = {2012-04-23 11:34:20 +0000},
	Date-Modified = {2012-04-23 11:34:20 +0000},
	Publisher = {Ceira - Coimbra},
	Title = {Arquitetura em Transi{\c c}{\~a}o},
	Year = {1965}}

@book{dewey1980,
	Address = {New York, NY, USA},
	Author = {John Dewey},
	Date-Added = {2012-04-23 11:34:16 +0000},
	Date-Modified = {2012-04-23 11:34:16 +0000},
	Publisher = {Perigee Books},
	Title = {Art as Experience},
	Year = {1980}}
